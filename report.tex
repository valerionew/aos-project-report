
\documentclass[10pt,a4]{article}
\usepackage[english]{babel}
\usepackage[utf8x]{inputenc}
\usepackage{amsmath}
\usepackage{graphicx}
\usepackage[colorinlistoftodos]{todonotes}
\usepackage{hyperref}
\usepackage{geometry}
\geometry{top=3cm,left=2cm,right=2cm,bottom=3cm}
\usepackage[scaled]{helvet}
\usepackage[T1]{fontenc}
\renewcommand\familydefault{\sfdefault}


\author{P. Rossi}
\date{\today}
\title{Project name}



\begin{document}
\maketitle
\tableofcontents


\section{Project data}

\begin{itemize}
\item 
  Project supervisor(s): put here the name of your supervisor

\item 
Describe in this table the group that is delivering this project:

\begin{center}
\begin{tabular}{lll}
Last and first name & Person code & Email address\\
\hline
  Foo Bar & 0101010 & foobar@example.com \\
  ... & &                     
\end{tabular}
\end{center}

\item
Describe here how development tasks have been subdivided among members
of the group, e.g.:

\begin{itemize}
\item Foo worked on the overall threading infrastructure
\item Bar worked on lockfree access etc..
\end{itemize}

\item Links to the project source code; Put here, if available, links to public repos hosting your project 

\end{itemize}


\section{Project description}

\textbf{2 pages max please}

\begin{itemize}
\item What is your project about?
\item Why it is important for the AOS course?
\end{itemize}

For those who choose to work on an open source project, please put here any
reference/copy of messages exchanged with project maintainers to \textbf{identify the subject
of the pull request}.

\subsection{Design and implementation}
Describe here the structure of the solution you devised. Note, don't put major
parts of the source code here; if you can, put hyperlinks to existing repos.

For those who choose to work on an open source project, please put here an
history (mail messages/github issues etc..) of the interaction with the
development team that helped you identify such design and the code reviews that
helped you improve it.

\section{Project outcomes}

\subsection{Concrete outcomes}
Describe the artifacts you've produced, if possible by linking to repo commits.
For those who choose to work on an open source project, please put here the 
\textbf{URL to your final pull request}.

\subsection{Learning outcomes}

What was the most important thing all the members have learned while
developing this part of the project, what questions remained unanswered,
how you will use what you've learned in your everyday life?
Please also indicate which tools you learned to use.

Examples:

\begin{itemize}
\item Foo learned to write multithreaded applications, he's probably going to
  create his own startup with what she has learned. She also learned how to
  debug with gdb.
\item Bar learned how to interact with the open source community, politely
  answering to code reviews and issuing pull requests through Git.
\end{itemize}

\subsection{Existing knowledge}
What courses you have followed (not only AOS) did help you in doing this project
and why? Do you have any suggestions on improving the AOS course with topics
that would have made it easier for you?

\subsection{Problems encountered}
What were the most important problems and issues you encountered? Did you ever
encountered them before? 

\begin{itemize}
\item Foo encountered a problem with some critical sections. He ended up
  rewriting existing lock implementation.
\end{itemize}

\section{Honor Pledge}
(\textbf{This part cannot be modified and it is mandatory to sign it})

I/We pledge that this work was fully and wholly completed within the criteria
established for academic integrity by Politecnico di Milano (Code of Ethics and
Conduct) and represents my/our original production, unless otherwise cited.

I/We also understand that this project, if successfully graded,  will fulfill part B requirement of the
Advanced Operating System course and that it will be considered valid up until
the AOS exam of Sept. 2022. 

\begin{flushright}
Group Students' signatures
\end{flushright}


\end{document}